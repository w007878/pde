\documentclass[9pt, a4paper]{article}

\usepackage{ctex}
\usepackage{amsmath}
\usepackage{amssymb}
\usepackage{geometry}
\usepackage{setspace}

\geometry{top=2.5cm, bottom=3cm, left=2.5cm, right=2.5cm}

\begin{document}

\title{微分方程作业}
\author{15336134 莫凡}
\maketitle

\section*{5.1.3}

根据截断误差的定义,有
\begin{spacing}{2.3}
\begin{equation}
	\begin{array}{rcl}
		R_j^k(u) & = & L_ku(x_j,t_k)-[Lu]_j^k\\
				 & = & \dfrac{u(x_j,t_{k+1})-u(x_j,t_{k-1})}{2\tau}-a\dfrac{u(x_{j+1},t_k)-u(x_j,t_{k+1})-u(x_j,t_{k-1})+u(x_{j-1},t_{k})}{h^2}\\
				 &   & - \left[\dfrac{\partial u(x_j,t_k)}{\partial t}-a\dfrac{\partial^2 u(x_j,t_k)}{\partial x^2}\right]
	\end{array}
\end{equation}
\end{spacing}

由泰勒展开可以得到

\begin{spacing}{2.3}
\begin{equation}
	\begin{array}{rcl}
		u(x_j,t) & = & u(x_j,t_k) + \dfrac{\partial u(x_j, t_k)}{\partial t}(t-t_k)\\
				 &   & + \dfrac{1}{2}\dfrac{\partial^2 u(x_j,t_k)}{\partial t^2}(t-t_k)^2
					   + \dfrac{1}{3!}\dfrac{\partial^3 u(x_j,t_k)}{\partial t^3}(t-t_k)^3\\
				 &   & + o\left((t-t_k)^4\right)
	\end{array}
\end{equation}
\end{spacing}

然后带入可以得到
\begin{spacing}{2.3}
\begin{equation}
	\begin{array}{rcl}
		\dfrac{u(x_j,t_{k+1})-u(x_j,t_{k-1})}{2\tau}&=&\dfrac{\partial u(x_j,t_k)}{\partial t}+\dfrac{\tau^2}{6}\dfrac{\partial^3u(x_j,t_k)}{\partial t^3}+O(\tau^3)\\
		& = & \dfrac{\partial u(x_j,t_k)}{\partial t}+O(\tau^2)
	\end{array}
\end{equation}
\end{spacing}

然后对另一维度进行泰勒展开

\begin{spacing}{2.3}
\begin{equation}
	\begin{array}{rcl}
		u(x,t_k) & = & u(x_j,t_k) + \dfrac{\partial u(x_j, t_k)}{\partial x}(x-x_j)\\
			   	 &   & + \dfrac{1}{2}\dfrac{\partial^2 u(x_j,t_k)}{\partial x^2}(x-x_j)^2+\dfrac{1}{3!}\dfrac{\partial^3 u(x_j,t_k)}{\partial x^3}(x-x_j)^3\\
				 &   & + O\left((x-x_j)^4\right)
	\end{array}
\end{equation}
\end{spacing}

带入式子的另一部分
\begin{spacing}{2.3}
\begin{equation}
	\begin{array}{ll}
		& \dfrac{u(x_{j+1},t_k)-u(x_j,t_{k+1})-u(x_j,t_{k-1})+u(x_{j-1},t_{k})}{h^2}\\
		=& \dfrac{\partial^2u(x_j,t_k)}{\partial x^2}+\dfrac{h^2}{2}\dfrac{\partial^4u(x_j,t_k)}{\partial x^4}+O(h^4)-\dfrac{\tau^2}{h^2}\dfrac{\partial^2u(x_j,t_k)}{\partial t^2}+O(\dfrac{\tau^4}{h^2})\\
		=& \dfrac{\partial^2u(x_j,t_k)}{\partial x^2}+O(h^2)+O(\dfrac{\tau^2}{h^2}) 
	\end{array}
\end{equation}
\end{spacing}

因此,总的截断误差带入得到
\begin{equation}
	R_j^k(u)=O(\tau^2)+O(h^2)+O(\dfrac{\tau^2}{h^2})
\end{equation}

\section*{5.1.4}
根据截断误差的定义,有
\begin{spacing}{2.3}
	\begin{equation}
	\begin{array}{rcl}
	R_j^k(u) & = & L_ku(x_j,t_k)-[Lu]_j^k\\
	& = & (1+\theta)\dfrac{u(x_j,t_{k+1})-u(x_j,t_k)}{\tau}-\theta \dfrac{u(x_j,t_k)-u(x_j,t_{k-1})}{\tau}\\
	&   & - \dfrac{a}{h^2}\left[u(x_{j+1},t_{k+1})-2u(x_j,t_{k+1})+u(x_{j-1},t_{k+1})\right]- \left[\dfrac{\partial u(x_j,t_k)}{\partial t}-a\dfrac{\partial^2 u(x_j,t_k)}{\partial x^2}\right]
	\end{array}
	\end{equation}
\end{spacing}
根据泰勒展开分别计算
\begin{spacing}{2.3}
	\begin{equation}
	\begin{array}{cl}
		& \dfrac{1}{\tau}[u(x_j,t_{k+1})-u(x_j,t_k)] \\
	  = & \dfrac{1}{\tau}[u(x_j,t_k)+\tau\dfrac{\partial u(x_j,t_k)}{\partial t}+\dfrac{\tau^2}{2!}\dfrac{\partial^2u(x_j,t_k)}{\partial t^2}+O(\tau^3)-u(x_j,t_k)]\\
	  = & \dfrac{\partial u(x_j,t_k)}{\partial t}+\dfrac{\tau}{2}\dfrac{\partial^2u(x_j,t_k)}{\partial t^2}+O(\tau^2)
	\end{array}
	\end{equation}
\end{spacing}
并且由上一题可以得到
\begin{spacing}{2.3}
	\begin{equation}
	\begin{array}{cl}
		 & \dfrac{1}{h^2}[u(x_{j+1},t_{k+1})-2u(x_j,t_{k+1})+u(x_{j-1},t_{k+1})]\\
		=& \dfrac{\partial^2u(x_j,t_k)}{\partial x^2}+\left(\dfrac{h^2}{12}+at\right)\dfrac{\partial^4u(x_j,t_k)}{\partial x^4}+O(\tau^2)+O(\tau^2h^2)+O(h^4)
	\end{array}
	\end{equation}
\end{spacing}
带入得到
\begin{spacing}{2.3}
	\begin{equation}
	R_j^k(u)=a[a\tau(\theta-\dfrac{1}{2})-\dfrac{h^2}{12}]\dfrac{\partial^4u(x_j,t_k)}{\partial x^4}+O(\tau^2)+O(\tau^2h^2)+O(h^4)
	\end{equation}
\end{spacing}

则当$a\tau(\theta-\dfrac{1}{2})-\dfrac{h^2}{12}=0$,即$\theta=\dfrac{1}{2}+\dfrac{1}{12r}$时,截断误差的阶最高为$O(\tau^2+h^4)$

\end{document}